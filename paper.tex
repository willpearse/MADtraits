\documentclass[12pt]{report}
\usepackage[utf8]{inputenc}
\renewcommand\thesection{\arabic{section}}
\usepackage[parfill]{parskip}
\usepackage{subcaption}
\usepackage{graphicx}
\usepackage[left=0.75in,right=0.75in,top=0.75in,bottom=0.75in]{geometry}
\usepackage{multirow}
\usepackage{hhline}
\usepackage{nth}
\usepackage{bm}
\usepackage{siunitx}
\usepackage{subcaption}
%\usepackage[doublespacing]{setspace}
%\usepackage{lineno} \linenumbers

\usepackage{xspace}
\newcommand{\GitHub}{\texttt{GitHub}\xspace}
\newcommand{\R}{\texttt{R}\xspace}
\newcommand{\natdb}{\texttt{NATDB}\xspace}

\usepackage{pifont}
\usepackage{minted}
\usepackage{hologo}

%%%%%%%%%%%%%%%%%%%%%%%%%%%%%%%%%%%
%Bibliography%%%%%%%%%%%%%%%%%%%%%%
%%%%%%%%%%%%%%%%%%%%%%%%%%%%%%%%%%%
\usepackage[citestyle=authoryear,bibstyle=authoryear,sorting=nyt,maxcitenames=2,maxbibnames=10,minbibnames=6,doi=false,url=false,isbn=false,firstinits=true,uniquename=false,uniquelist=false]{biblatex}
%Italicise et al.
\renewbibmacro*{name:andothers}{
% Based on name:andothers from biblatex.def
\ifboolexpr{ test{\ifnumequal{\value{listcount}}{\value{liststop}}} and test \ifmorenames } {\ifnumgreater{\value{liststop}}{1} {\finalandcomma}
{}%
\andothersdelim\bibstring[\emph]{andothers}} {}}
%Make 'and' an ampersand
\renewcommand*{\finalnamedelim}{%
\ifnumgreater{\value{liststop}}{2}{\finalandcomma}{}%
\addspace\&\space}
%Remove 'in:'
\renewbibmacro{in:}{}
%Remove pp. on pages numbers
\DeclareFieldFormat[article]{pages}{#1}%
%Remove unwanted entries
\AtEveryBibitem{%
  \clearfield{day}%
  \clearfield{month}%
  \clearfield{endday}%
  \clearfield{endmonth}%
  \clearfield{url}%
  \clearfield{URL}%
  \clearfield{eprint}%
}
%Remove spaces between initials
%Remove double quotes around titles
\DeclareFieldFormat*{citetitle}{#1}
\DeclareFieldFormat*{title}{#1}
\addbibresource{/home/will/Papers/Library.bib}
%%%%%%%%%%%%%%%%%%%%%%%%%%%%%%%%%%%

\begin{document}

\title{\natdb : An \R package that downloads species trait data, but
  is Not A Trait DataBase} \author{Order TBD: William D.\ Pearse,
  Konrad Hafen, Mallory Hagadorn,\\ Marley Haupt, Spencer B.\ Hudson,
  Sylvia Kinosian, Ryan McCleary,\\ Alexandre Rego, \& Katie Welgarz}
\date{\today}
\maketitle

\section{Abstract}
\begin{enumerate}
\item Ecologists and evolutionary biologists often wish to make use of
  species trait data, either as ancillary data, such as in community
  ecology, or as the primary focus of a study, such as
  macro-evolutionary modelling.
\item Such biologists are often hampered by the difficulties of
  collecting sufficient trait data from published sources. There are
  very few open access databases of species' traits.
\item We present \natdb, an \R package---not a trait database---that
  automatically downloads species trait data from existing sources.
\item NATDB collates trait data from over XXX publications across XXX
  species, and at the time of writing downloads over XXX individual
  trait measurements.
\item NATDB is emphatically \emph{not} a trait database: it
  circumvents issues over intellectual ownership of species trait
  data by not distributing data, but rather giving users automated
  tools to build their own data from existing, published datasets. Our
  hope is to establish a user community around this package, adding
  both additional data sources and cleaning routines for the data
  itself.
\item Upon acceptance, \natdb will be uploaded to CRAN, but is
  currently available for download on \GitHub. It can be installed by
  typing
  \texttt{library(devtools);install\_github("willpearse/natdb")} at an
  \R console.
\end{enumerate}

\section{Introduction}
Ecologists and evolutionary biologists have long recognised the
importance of (functional) traits in their work
\autocite{Diaz2001}. Large datasets of plants \autocite{Kattge2011},
mammals \autocite{Jones2009}, and birds \autocite{Wilman2014} have
opened the door to analyses of the evolution \autocite{Harmon2010} and
global distribution \autocite{Kattge2011} of trait diversity.
% This last clause is weak; find more evidence.
Species' traits help us better predict how species will respond to
land use \autocite{Mayfield2010a} and climate change
\autocite{Estrada2016}, allowing us to generalise and compare across
species to find general biodiversity patterns. 
% Blimey this is thin on the ground!

Yet, despite its importance, it is often difficult to find data on
species' functional traits. We suggest there are three main reasons for
this: (1) it is difficult to obtain trait data, (2) it is difficult to
collate trait data, and (3) concerns have been raised about
intellectual property and the distribution of trait data. (1) Often
the most functionally important species traits are the most difficult
to measure \autocite{Cornelissen2003,Violle2007}, and even when
measuring a trait is simple finding a specimen is often not. Usefully
measuring and defining species' traits is not an easy thing. (2)
Creating and maintaining large databases is difficult: the
nomenclature for species and traits is not universal
\autocite{Kattge2011,Hudson2017}, and unifying concepts across
different datasets takes detailed knowledge of species and their
traits. (3) Unlike other other kinds of data such as DNA sequences
\autocite{Benson2013}, the publication of species trait data has been
controversial \autocite[\emph{e.g.},][]{Poisot2014,Moles2013}.
% Are those citations right? I thought Moles was a reply to Poisot...
The reasons for this are complex and numerous, but perhaps the most
compelling argument is a concern that releasing data will lead to it
being `hoovered up' into a database where the creator of the database
will get credit but the original collectors of the data none.

We present here \natdb, an \R package that releases over XXX pieces of
trait data for over XXX species, making existing species trait data
more widely available for use by ecologists and evolutionary
biologists. We argue that \natdb is a prototype for a new way of
making data more accessible that avoids concerns about data ownership:
\natdb is \emph{not a trait database}. \natdb is a software package
that simplifies the process of collating data the user already had
access to, and so obviates any concerns over `hoovering up' data
because it simply retrieves data the authors have already publicly
released. \natdb contains no data, and so users must cite the sources
of data when using the package. This model both liberates the vast
trait-based knowledge that already exists in the literature, and
protects the intellectual contributions of those who collected the
data in the first place.

\section{Description}
\natdb consists of a series of internal functions, each of which
downloads species trait data from a single paper. Typically, a user
will download a set of data and then subset that down to only the
species or traits that they require. Note that, by default, \natdb
waits ten seconds between downloading datasets to minimise impact on
journals' servers. For example, the following would download all the
data in \natdb, and then subset that down to only two kinds of traits
for two species:

\begin{minted}{splus}
  library(natdb)
  data <- natdb(taxon)
  species <- c("Quercus_robur", "Pinus_sylvestris")
  traits <- c("SLA", "height")
  subset.data <- data[species, traits]
\end{minted}

By default, \natdb caches whatever it has been asked to download
during an \R session. So, for example, if the user were to realise
that they wanted data on an additional species or trait after
executing the code above, they could run the entire script again and
\natdb would not download any more data. The \texttt{cache} option
allows the user to override this behaviour if they desire. A user can
also specify a directory when invoking \natdb so that they can save
their searches between sessions. The following code, for example,
would cache results between sessions, and would add additional data to
that cache as new versions of \natdb were released. This is the
recommended way to use \natdb, as it saves the user time and reduces
server load.

\begin{minted}{splus}
  data <- natdb(cache="~/.natdb_cache/")
  subset.data <- data[c("Phocoena_phocoena","Tursiops_truncatus"),]
\end{minted}

\natdb has a single class for dealing with trait data, called
(unimaginatively) \natdb. This class has \texttt{head},
\texttt{print}, \texttt{summary}, \texttt{rbind}, \texttt{cbind}, and
\texttt{as.data.frame} methods to make it easier for the user to work
with their data. Internally, \natdb distinguishes between, and convert
all data into, \texttt{numeric} and \texttt{character} types, and
\emph{melts} \autocite[\emph{sensu}][]{Wickham2007} all data within
these types. This makes it easy to add new data to an existing \natdb
object, while keeping reducing memory load. If \natdb were to store
data in a \texttt{data.frame}-like format, it would require XXX cells
(XXX species, XXX traits) to store all its data, XXX\% of which would
be missing.

Ready access to meta-data is important in any database. The databases
\natdb can build are complex, in that the meta-data that different
source datasets provide can vary a great deal. We follow the general
approach of \emph{FigShare} (\url{https://figshare.com/}) and
\emph{DataDryad} (\url{http://datadryad.org/}) in not enforcing rigid
meta-data requirements, but placing the meta-data of each source
dataset within a comparable framework so as to allow users to
interrogate the meta-data in their own way. Thus while we do
standardise some aspects of the data (\emph{e.g.}, ensuring all
latitude and longitude measurements, where present, are termed
\texttt{latitude} and \texttt{longitude}), users must check whatever
subset of data they have to see what meta-data are available. For
example:

\begin{minted}{splus}
  head(metadata(subset.data))
  table(metadata(subset.data))
  plot(subset.data$SLA ~ metadata(subset.data)$latitude)
\end{minted}

We make no guarantee that the taxonomy or units of the data within
\natdb are internally compatible: users are responsible for checking
the validity of the data they have collated. However, we have written
wrappers for common taxonomic services for cleaning problems with
species nomenclature using existing \R packages
\autocite[\texttt{Taxonstand} and others?]{Cayuela2012}.

\begin{minted}{splus}
  data <- natdb(taxon)[c("Quercus_robur", "Pinus_sylvestris"), c("SLA", "height")]
  subset.data <- data[units(data) == "mm", "height"]
  subset.data <- tpl.clean(subset.data)
\end{minted}

Finally, it is important that those who generated the data \natdb
downloads are appropriately cited. It is simple to generate
\hologo{BibTeX}, RIS, and plaintext files to help with citations:

\begin{minted}{splus}
  citation(subset.data, "bibtex")
\end{minted}


\begin{table}
  \begin{tabular}{lllll}
    Taxonomic group & \# Species & \# Traits & \% Complete & Citations \\ \hline
    Plants & & & \textcite{Wright2004}\\
    Mammals & & & \textcite{Jones2009,Wilman2014}\\
    Birds & & & \textcite{Wilman2014}\\
    ...TBC... \\ \hline
  \end{tabular}
  \caption{Overview of data available for download within
    \natdb. Overall, the package downloads XXX data points, covering
    XXX species and XXX separate functional traits. XXX\% of these
    trait values have some form of meta-data associated with them.}
  \label{overview}
\end{table}

\section{Comparison with existing tools}
As table \ref{comparison} shows, \natdb downloads more data than a set
of comparable databases, although its data is, perhaps by nature of
its wide taxonomic coverage, less complete per species. The most
important way in which \natdb differs from the other tools and
datasets in table \ref{comparison} is that it has been designed, from
the ground-up, to be easy to extend. Adding a publication's data to
the package requires no knowledge other than the basic structure of
data to be added. The average length of the functions that load a data
structure into \natdb is XXX lines of \R code, in part because as part
of this project we developed code for the \R package \texttt{fulltext}
\autocite{Chamberlain2015} to automate the download of data from
published papers. \natdb uses reflective programming to query itself
to determine what datasets are available for download, and as such
extension is trivial. This represents a major advantage to \natdb: it
is a living package that will, we hope, grow as authors add their own
publications to it. We provide detailed instructions on how to
contribute data sources to \natdb in the package's vignette.

% Everyone: this paragraph is important. People work very hard on
% these databases and we don't want to be glib! Some of these people
% are also my friends!... :D
The flexibility and scope of \natdb, however, means it has not been as
carefully cleaned and checked as datasets typically are. This is by
design: \natdb is fundamentally different, and we use TRY
\autocite{Kattge2011} to illustrate this. TRY is a carefully-collated
dataset that has required thousand of person-hours to create, and to
reflect this and ensure that the data is used correctly, its authors
require that many data contributors and the two lead authors of the
database are offered co-authorship on any publication making use of
TRY data. We consider this a reasonable request given the amount of
effort involved in producing a database like TRY, and the feedback and
data-validation that these additional co-authors provide to a finished
manuscript. \natdb is not a database and does not follow this model:
the data are provided `as-is' and neither we, nor the original data
publishers, require co-authorship for use of the package. Basic
taxonomic and data %...talk about how some basic cleaning, but nothing
                   %fancy

\begin{table}
  \begin{tabular}{lllllll}
    Dataset & \R native? & Taxonomic scope & \# Species & \# Traits & \% Complete & Citation \\ \hline
    TRY & \ding{55} & Plants & & & \textcite{Kattge2011} \\
    D3 & \ding{55} & Plants & & & \textcite{Hintze2013} \\
    TR8 & \ding{51} & Plants & & & \textcite{Gionata2015} \\ \hline
    \natdb & \ding{51} & Organisms  \\ \hline
  \end{tabular}
  \caption{Comparison of \natdb to existing packages or databases. As
    described in the text, we only compare \natdb with open access
    databases and packages.}
  \label{comparison}
\end{table}

\section{Future directions}
We actively encourage code contributions, and the package's online
vignette contains a detailed set of instructions on how to contribute
functions that download data from new sources. Our intention is to
make the process as simple as possible, in the hope that authors who
release the data underlying their publications will contribute to the
package, in the process making \natdb better and making it easier for
others to use (and cite) their work. This, in part, is the reason we
have few formal checks on meta-data and units within \natdb: we
consider the first hurdle to be getting more data into a useable
format within \R, and all other issues can be handled later. We hope
that, using \natdb as a base, others will develop cleaning and
checking routines that can be applied to the package. Whether these
will be incorporated into \natdb itself, or released as separate
companion package(s), remains to be seen.

\natdb is both an experiment in a new way of making data more
accessible, and a useful resource that we are already making use of in
our daily working lives. It is our hope that \natdb, and resources
like it, will continue to grow.
% This ending is nauseating

\end{document}
%%% Local Variables:
%%% mode: latex
%%% TeX-master: t
%%% End:
